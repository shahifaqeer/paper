\section{Introduction}

Home networks contain a variety of heterogeneous devices competing over the wireless channel for Internet access time. Increasing user demands and ease of accessing content on such wireless devices have resulted in 802.11n, offering datarates as high as 300Mbps. Similarly, there has also been a steady increase in access link capacities to cater to users' high bandwidth requirements. With such high rates many users still face performance issues in home networks, which cannot be pinpointed by simplistic end host bandwidth measurement tools that were not meant for the wireless medium \cite{speedtest}, \cite{netalyzr}.

In our work, we develop an end-host based approach to identify the bottleneck link between the home or on the access network. This information allows ISPs and users to identify whether the bottleneck issue is local or on the access side. Furthermore, content providers such as video hosts (e.g., Netflix) can use this tool to better understand whether lower performance is caused due to local wireless network issues, or because if there was traffic shaping or congestion at the access network side.

Previous research in bottleneck identification has mainly concentrated on tomography \cite{tools}, which were not designed to work on the wireless medium due to its non-work conserving nature. Whereas to study and diagnose the home wireless network, researchers require extra monitoring nodes and kernel changes on the gateway to get an in-depth view into wireless parameters.

A promising method is using collaborative devices, such as the gateway \cite{wtf}, or another wired node in the home \cite{wlanprobe}, to identify the bottleneck. But this approach still requires collaboration with either a highly capable OpenWRT gateway, or a wired computer in the home, both of which are not common current state of the art in home networks. Thus this work has not been tested and evaluated on a large scale. In our work, we progress in this direction by directing the user to use such collaborative measures with other wireless devices available in the home, such as a smartphone.
%Previous research in the domain of bottleneck link identification can be classified into three types.

%\textbf{Tomography:} using packet pair techniques to find the faulty link in an access network based on packet timings at the receiver host \cite{tools}. This approach has not been successfully applied over home networks as they usually contain a wireless links, which are non-work conservative, thereby making timing based approaches statistically prone to error.

%\textbf{Promiscuous nodes or extra antennas:} in the vicinity of the gateway which give a detailed view of the wireless mediums utilization and allow monitoring of radiotap statistics \cite{wise}. These works may also require certain kernel modifications in the gateways to collect erroneous packets. This approach may work well for enterprise networks or laboratories, but is generally impractical and unscalable for home networks, which usually contain just one simple gateway device. 

%\textbf{Gateway collaboration:} allows traffic monitoring from custom firmwares running on home routers and diagnose bottlenecks via passive measurements \cite{wtf}. This approach solves the bottleneck detection problem and has the additional advantage of not introducing any active probing traffic, but it hasn’t been evaluated at scale yet, and requires root access to a capable gateway device inside the home. Similarly, by having a wired device attached to the gateway, one can perform active measurements to diagnose wireless performance by studying delays, as the wired link to the gateway is considered to never be the bottleneck link \cite{wlanprobe}. This approach is the closest to our collaborative solution but requires a wired device collaboration running custom tools inside the home, which is a rare occurrence in today's primarily wireless home networks.