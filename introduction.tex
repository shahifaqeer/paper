\section{Introduction}

%Why we need such a tool - motivation
Home networks contain a variety of heterogeneous devices competing over the wireless channel for Internet access time. Increasing user demands and ease of accessing content have resulted in the development of high speed WiFi and high rate of access links. However, users still face performance issues in home networks, which cannot be pinpointed by end host speed tests. ISPs require a tool to infer the most likely location of a performance bottleneck - home or access. Furthermore, such information will allow content providers such as video host services to understand if a throughput bottleneck in their end to end flow is due to wireless network issues inside the home, or due to congestion or traffic shaping on the access link. Our work presents a first step in this direction, before delving into the reasons for the bottleneck issue, as both wireless network performance and access link issues require different diagnostic approaches and expertise.

%Increasing user demands and ease of accessing content on such wireless devices have resulted in 802.11n, offering datarates as high as 300Mbps. Similarly, there has also been a steady increase in access link capacities to cater to users' high bandwidth requirements. With such high rates many users still face performance issues in home networks, which cannot be pinpointed by simplistic end host bandwidth measurement tools that were not meant for the wireless medium \cite{speedtest}, \cite{netalyzr}.

%In our work, we develop an end-host based approach to identify the bottleneck link between the home or on the access network. This information allows ISPs and users to identify whether the bottleneck issue is local or on the access side. Furthermore, content providers such as video hosts (e.g., Netflix) can use this tool to better understand whether lower performance is caused due to local wireless network issues, or because if there was traffic shaping or congestion at the access network side.

%Why existing doesn't work
Existing bottleneck detection solutions and wireless performance analysis tools usually require external hardware and significant changes to protocols, making them practically non-scalable across home networks. Our proposed solution works under these constraints, and is practically scalable across heterogeneous home networks.

%What we do
We develop a collaborative tool that runs on end host devices in the home to locate the bottleneck link. Our approach takes advantage of multiple viewpoints offered by multiple home network devices measurements. By collaborating between devices to perform active measurements, we can draw inferences about each individual link, which indicates whether the bottleneck link is inside the home or on the access link.

%why browser extension
To ensure that our collaborative measurement algorithm is deployable in the variety of heterogeneous devices present in homes, we deploy it as a browser extension. This makes our tool scalable and easily implementable. However this approach also introduces some challenges due to the limited capability of the browser environment.
