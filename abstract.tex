\begin{abstract}

Identifying the performance bottleneck in a home network allows for better diagnosis by ISPs, users, and content providers. Existing solutions either troubleshoot network faults, or analyze wireless performance using multiple promiscuous gateways, making them impractical and unscalable in a home network environment consisting of simplistic gateways and wireless devices. Our work presents a highly scalable tool to locate the bandwidth bottleneck (thin link) in a home network from the perspective of the end device. We propose an optimized active measurement approach to estimate the available bandwidth of the wireless (home) and the wireline (access) link by either coordinating with the access point for measurements, or collaborating with other home network devices. Our approach uses bandwidth measurements and timing information from active traffic, and requires no protocol modification, to detect bottlenecks in both wireless network and access link. We evaluate our tool by simulation, on a controlled testbed, and in real home networks. Preliminary results show that our one-shot measurement strategy can accurately detect the bottleneck link within a certain range depending on the variation of wireless and access link bandwidth measurements, either with gateway coordination or by collaborating with another generic device inside the home.

\end{abstract}
