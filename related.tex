\section{Realted work}
Per-flow Qos for home and small office broadband Internet access has not been extensively investigated by the network research community. In fact, only two approaches, to provide end to end QoS for IP networks, have been proposed:  integrated services and differentiated services. DiffServ is used in core networks and IntServ in edge networks. Thanks to the flexibility provided by SDN , many researchers worked on providing QoS for different types of applications that depends on the application's needs.\\
Sonkoly et al. \cite{SonkolyOfelia2012} proposed an architectural extension to  the main European OpenFlow experimental testbed with integrated and manageable QoS support. The goal is to make this testbed is capable of running QoS related experiments.
Kim et al. \cite{KimscalableQoS2010} presented a solution most closely related to ours, although different in the approach goals. The authors claimed that their controller is the setting of rate limiters at the edge switches and priority queues for flow at each path hop. They proposed a QoS control framework for automated fine-grained management of OpenFlow networks. In our approach we provide the same automated traffic shaping but for one single gateway while all the traffic sources and types are identified.\\
Georgopoulos et al. \cite{GeorgopoulosQOE2013} proposed a OpenFlow-assisted Framework that fairly maximises users' QoE in Home networks for multimedia flow. The idea is to dynamically adjust multimedia flows characteristics to ensure QoE fairness. The authors addressed the issue of allocating the network resource to each device without considering how these flows are classified into multimedia flows.\\
Mortier et al.  \cite{MortierHomeInterface2011} proposed a home networking platform that provides per-flow measurement and management capabilities supporting. The home users can monitors per-protocol bandwidth consumption for each device and can also limits the  access to specified web-hosted services.
\textcolor{red}{ Said ! Mention these works and their limitations \cite{IshimoriQoSFlow2012} 
\cite{KoOpenQFlow2013} 
}

