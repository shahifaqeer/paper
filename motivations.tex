\section{Research motivations}
A fundamental task of access routers is to fairly allocate bandwidth to multiple end-users. Routers today support the single class of best-effort service that the current Internet provides. They decide which packets to drop when they becomes congested according to the arrival orders. OpenFlow switches offer the flexibility to integrate new functionality, allowing end-users to specify which flow to drop. Such a mechanism allows certain applications to use the maximum throughput available when it is needed.

The following example illustrates a need for rate limiting traffic for multiple applications over IP. Consider a real-estate agency where employees use Voice-over-IP telephony (VoIP) to make phone calls over an IP data network. They are also using their network to meet their day-to-day business needs such as producing numerous daily reports. When an employee starts watching a video over the Internet and downloading an OS update during work hours, it can affect the bandwidth available to all other employees in the office. In this case, rate limiting for specific application would offer a better Quality of Experience for employees and help them to improve the performance of their network. 

Consider another example where a user wants to watch a movie from Netflix on an HD TV and does not want other home network traffic to affect the quality of the video stream. In this case, the end-user needs to manage the bandwidth and give a greater portion to his Netflix video stream. Unfortunately, today's residential and small business networks have no mechanism for providing rate limiting for different services over IP. 

To address this issue, we propose a SDN approach that helps the end-user to define his preference to multiple internet services. A primary motivation of this work is to provide the gateway with the ability to identify the traffic flows and to enable forwarding rules on how they should be treated. There is a growing need to support the rapid deployment of the real-time multimedia applications in broadband access networks. Thus, deploying per-flow QoS will allow end-users to specify policies (like rate limiting and bandwidth allocation) to achieve an overall better quality of experience.

\begin{figure}[!ht]
\begin{center}
    \includegraphics[width=3.2in,height=1.7in]{architecture2.png}
\end{center}
\caption{An architecture of the proposed solution for home networks.}
\label{fig:sim}
\end{figure}

This consists of a dual Open vSwitch topology contained entirely within the home router. The Eastbound switch maintains Internet connectivity and the Westbound switch connects with the home devices. Each interfaces between the switches carries traffic for a particular application class ({\em i.e.}, voice, video etc). This configuration enables the use of tc NEED CITATION HERE on these links to provide bandwidth limitation. This mechanism aims to provide a quantitative bandwidth guarantee to some services; such allocations are performed based on the policies defined by the user. The rate limiter component is responsible for controlling the rate of traffic received by a given Internet service such as VoD, VoIP, etc. Figure \ref{fig:sim} depicts the architecture of the solution described deployed in a home network.